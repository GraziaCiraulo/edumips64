\batchmode
\documentclass[12pt]{report}
\RequirePackage{ifthen}


\usepackage{verbatim}
\usepackage{amsfonts}
\usepackage{indentfirst}
\usepackage{makeidx}
\usepackage{listings}
\usepackage{fancyhdr}
\usepackage[italian]{babel}%
\providecommand{\EM}{EduMIPS64}%
\providecommand{\EV}{0.5}%
\providecommand{\MS}{MIPS64}%
\providecommand{\SC}{\texttt{SYSCALL}}%
\providecommand{\WARN}{\textbf{ATTENZIONE:} Questa \`{e} un'istruzione deprecated, che
non appartiene al repertorio delle istruzioni MIPS64{}, ma \`{e} stata inclusa nel set di
istruzioni per mantenere la compatibilit\`{a} con altri simulatori.}%
\providecommand{\MISN}{\textbf{NB:} La 'U' dell'istruzione sta per 'unsigned'; tuttavia questo \`{e} 
un nome equivoco, perch\'{e} l'operazione considera i valori con segno. La sua caratteristica \`{e} 
quella di non lanciare eccezioni (nemmeno in caso di overflow).} 


\lstdefinelanguage{MIPS64}%
{keywords={add, addi, addiu, addu, and, andi, b, beq, beqz, bgez, bne, bnez, break,% 
 bubble, dadd, daddi, daddu, daddiu, daddui, ddiv, ddivu, div, divu, dmult, dmultu,% 
 dsll, dsllv, dsra, dsrav, dsrl, dsrlv, dsub, dubu, halt, j, jal, jalr, jr, lb, lbu, lh, lhu, lw, lwu,%
 ld, lui, mfhi, mflo, movn, movz, mult, multu, nop, or, ori, sb, sh, sw, sd, sll, sllv, slt, slti,%
 sltiu, sltu, sra, srav, srl, srlv, sub, subu, syscall, trap, xor, xori},%
 comment=[l];,%
 directives={.data, .text, .ascii, .asciiz, .space, .byte, .half, .word, .word32, .word64, .double}%
}[keywords,comments,strings,directives]%



\DeclareFontShape{OT1}{cmtt}{bx}{n}{ <5><6><7><8><9><10><10.95><12><14.4><17.28><20.74><24.88>cmttb10}{} 


\lstdefinestyle{mips}{
	language=MIPS64,
	numbers=left,
	basicstyle=\small ,
	keywordstyle=\bfseries ,
	identifierstyle=\ttfamily ,
	stringstyle=\ttfamily ,
	captionpos=b,
	abovecaptionskip=15pt,
	showstringspaces=false
}
\lstdefinestyle{java}{
	language=java,
	numbers=left,
	basicstyle=\small ,
	keywordstyle=\bfseries ,
	identifierstyle=\ttfamily ,
	stringstyle=\ttfamily ,
	captionpos=b,
	abovecaptionskip=15pt,
	showstringspaces=false
}


\usepackage{subfigure}
\usepackage[top=5.3cm, bottom=2cm, left=4cm, right=3cm]{geometry}


\makeindex


\author{Andrea Spadaccini\\Traduzione italiana di Simona Ullo}
\title{Manuale Utente EduMIPS64 0.5}



\hyphenation{mo-di-fica} 

\hyphenation{vi-sua-liz-zan-do} 

\hyphenation{dia-lo-go} 




\usepackage[dvips]{color}


\pagecolor[gray]{.7}

\usepackage[latin1]{inputenc}



\makeatletter

\makeatletter
\count@=\the\catcode`\_ \catcode`\_=8 
\newenvironment{tex2html_wrap}{}{}%
\catcode`\<=12\catcode`\_=\count@
\newcommand{\providedcommand}[1]{\expandafter\providecommand\csname #1\endcsname}%
\newcommand{\renewedcommand}[1]{\expandafter\providecommand\csname #1\endcsname{}%
  \expandafter\renewcommand\csname #1\endcsname}%
\newcommand{\newedenvironment}[1]{\newenvironment{#1}{}{}\renewenvironment{#1}}%
\let\newedcommand\renewedcommand
\let\renewedenvironment\newedenvironment
\makeatother
\let\mathon=$
\let\mathoff=$
\ifx\AtBeginDocument\undefined \newcommand{\AtBeginDocument}[1]{}\fi
\newbox\sizebox
\setlength{\hoffset}{0pt}\setlength{\voffset}{0pt}
\addtolength{\textheight}{\footskip}\setlength{\footskip}{0pt}
\addtolength{\textheight}{\topmargin}\setlength{\topmargin}{0pt}
\addtolength{\textheight}{\headheight}\setlength{\headheight}{0pt}
\addtolength{\textheight}{\headsep}\setlength{\headsep}{0pt}
\setlength{\textwidth}{349pt}
\newwrite\lthtmlwrite
\makeatletter
\let\realnormalsize=\normalsize
\global\topskip=2sp
\def\preveqno{}\let\real@float=\@float \let\realend@float=\end@float
\def\@float{\let\@savefreelist\@freelist\real@float}
\def\liih@math{\ifmmode$\else\bad@math\fi}
\def\end@float{\realend@float\global\let\@freelist\@savefreelist}
\let\real@dbflt=\@dbflt \let\end@dblfloat=\end@float
\let\@largefloatcheck=\relax
\let\if@boxedmulticols=\iftrue
\def\@dbflt{\let\@savefreelist\@freelist\real@dbflt}
\def\adjustnormalsize{\def\normalsize{\mathsurround=0pt \realnormalsize
 \parindent=0pt\abovedisplayskip=0pt\belowdisplayskip=0pt}%
 \def\phantompar{\csname par\endcsname}\normalsize}%
\def\lthtmltypeout#1{{\let\protect\string \immediate\write\lthtmlwrite{#1}}}%
\newcommand\lthtmlhboxmathA{\adjustnormalsize\setbox\sizebox=\hbox\bgroup\kern.05em }%
\newcommand\lthtmlhboxmathB{\adjustnormalsize\setbox\sizebox=\hbox to\hsize\bgroup\hfill }%
\newcommand\lthtmlvboxmathA{\adjustnormalsize\setbox\sizebox=\vbox\bgroup %
 \let\ifinner=\iffalse \let\)\liih@math }%
\newcommand\lthtmlboxmathZ{\@next\next\@currlist{}{\def\next{\voidb@x}}%
 \expandafter\box\next\egroup}%
\newcommand\lthtmlmathtype[1]{\gdef\lthtmlmathenv{#1}}%
\newcommand\lthtmllogmath{\dimen0\ht\sizebox \advance\dimen0\dp\sizebox
  \ifdim\dimen0>.95\vsize
   \lthtmltypeout{%
*** image for \lthtmlmathenv\space is too tall at \the\dimen0, reducing to .95 vsize ***}%
   \ht\sizebox.95\vsize \dp\sizebox\z@ \fi
  \lthtmltypeout{l2hSize %
:\lthtmlmathenv:\the\ht\sizebox::\the\dp\sizebox::\the\wd\sizebox.\preveqno}}%
\newcommand\lthtmlfigureA[1]{\let\@savefreelist\@freelist
       \lthtmlmathtype{#1}\lthtmlvboxmathA}%
\newcommand\lthtmlpictureA{\bgroup\catcode`\_=8 \lthtmlpictureB}%
\newcommand\lthtmlpictureB[1]{\lthtmlmathtype{#1}\egroup
       \let\@savefreelist\@freelist \lthtmlhboxmathB}%
\newcommand\lthtmlpictureZ[1]{\hfill\lthtmlfigureZ}%
\newcommand\lthtmlfigureZ{\lthtmlboxmathZ\lthtmllogmath\copy\sizebox
       \global\let\@freelist\@savefreelist}%
\newcommand\lthtmldisplayA{\bgroup\catcode`\_=8 \lthtmldisplayAi}%
\newcommand\lthtmldisplayAi[1]{\lthtmlmathtype{#1}\egroup\lthtmlvboxmathA}%
\newcommand\lthtmldisplayB[1]{\edef\preveqno{(\theequation)}%
  \lthtmldisplayA{#1}\let\@eqnnum\relax}%
\newcommand\lthtmldisplayZ{\lthtmlboxmathZ\lthtmllogmath\lthtmlsetmath}%
\newcommand\lthtmlinlinemathA{\bgroup\catcode`\_=8 \lthtmlinlinemathB}
\newcommand\lthtmlinlinemathB[1]{\lthtmlmathtype{#1}\egroup\lthtmlhboxmathA
  \vrule height1.5ex width0pt }%
\newcommand\lthtmlinlineA{\bgroup\catcode`\_=8 \lthtmlinlineB}%
\newcommand\lthtmlinlineB[1]{\lthtmlmathtype{#1}\egroup\lthtmlhboxmathA}%
\newcommand\lthtmlinlineZ{\egroup\expandafter\ifdim\dp\sizebox>0pt %
  \expandafter\centerinlinemath\fi\lthtmllogmath\lthtmlsetinline}
\newcommand\lthtmlinlinemathZ{\egroup\expandafter\ifdim\dp\sizebox>0pt %
  \expandafter\centerinlinemath\fi\lthtmllogmath\lthtmlsetmath}
\newcommand\lthtmlindisplaymathZ{\egroup %
  \centerinlinemath\lthtmllogmath\lthtmlsetmath}
\def\lthtmlsetinline{\hbox{\vrule width.1em \vtop{\vbox{%
  \kern.1em\copy\sizebox}\ifdim\dp\sizebox>0pt\kern.1em\else\kern.3pt\fi
  \ifdim\hsize>\wd\sizebox \hrule depth1pt\fi}}}
\def\lthtmlsetmath{\hbox{\vrule width.1em\kern-.05em\vtop{\vbox{%
  \kern.1em\kern0.8 pt\hbox{\hglue.17em\copy\sizebox\hglue0.8 pt}}\kern.3pt%
  \ifdim\dp\sizebox>0pt\kern.1em\fi \kern0.8 pt%
  \ifdim\hsize>\wd\sizebox \hrule depth1pt\fi}}}
\def\centerinlinemath{%
  \dimen1=\ifdim\ht\sizebox<\dp\sizebox \dp\sizebox\else\ht\sizebox\fi
  \advance\dimen1by.5pt \vrule width0pt height\dimen1 depth\dimen1 
 \dp\sizebox=\dimen1\ht\sizebox=\dimen1\relax}

\def\lthtmlcheckvsize{\ifdim\ht\sizebox<\vsize 
  \ifdim\wd\sizebox<\hsize\expandafter\hfill\fi \expandafter\vfill
  \else\expandafter\vss\fi}%
\providecommand{\selectlanguage}[1]{}%
\makeatletter \tracingstats = 1 


\begin{document}
\pagestyle{empty}\thispagestyle{empty}\lthtmltypeout{}%
\lthtmltypeout{latex2htmlLength hsize=\the\hsize}\lthtmltypeout{}%
\lthtmltypeout{latex2htmlLength vsize=\the\vsize}\lthtmltypeout{}%
\lthtmltypeout{latex2htmlLength hoffset=\the\hoffset}\lthtmltypeout{}%
\lthtmltypeout{latex2htmlLength voffset=\the\voffset}\lthtmltypeout{}%
\lthtmltypeout{latex2htmlLength topmargin=\the\topmargin}\lthtmltypeout{}%
\lthtmltypeout{latex2htmlLength topskip=\the\topskip}\lthtmltypeout{}%
\lthtmltypeout{latex2htmlLength headheight=\the\headheight}\lthtmltypeout{}%
\lthtmltypeout{latex2htmlLength headsep=\the\headsep}\lthtmltypeout{}%
\lthtmltypeout{latex2htmlLength parskip=\the\parskip}\lthtmltypeout{}%
\lthtmltypeout{latex2htmlLength oddsidemargin=\the\oddsidemargin}\lthtmltypeout{}%
\makeatletter
\if@twoside\lthtmltypeout{latex2htmlLength evensidemargin=\the\evensidemargin}%
\else\lthtmltypeout{latex2htmlLength evensidemargin=\the\oddsidemargin}\fi%
\lthtmltypeout{}%
\makeatother
\setcounter{page}{1}
\onecolumn

% !!! IMAGES START HERE !!!



\setlength{\baselineskip}{1.60\baselineskip}%

\setlength{\baselineskip}{1.60\baselineskip}


\newlength{\centeroffset}%



\setlength{\centeroffset}{-0.5\oddsidemargin}%

\setlength{\centeroffset}{-0.5\oddsidemargin}


\addtolength{\centeroffset}{0.5\evensidemargin}%

\stepcounter{chapter}
\stepcounter{section}
{\newpage\clearpage
\lthtmlfigureA{lstlisting100}%
\begin{lstlisting}[caption={Esempio di codice EduMIPS64{}}, label={code:sample}, style={mips}]
; Questo e' un commento
        .data
label:  .word   15     ; Questo e' un commento in linea
\par
.code
        daddi   r1, r0, 0
        syscall 0
\end{lstlisting}%
\lthtmlfigureZ
\lthtmlcheckvsize\clearpage}

\stepcounter{subsection}
{\newpage\clearpage
\lthtmlfigureA{lstlisting153}%
\begin{lstlisting}[caption={Byte adiacenti}, label={code:data_array}, style={mips}]
	.data
	.byte	1, 2, 3, 4
	.byte	1
	.byte	2
	.byte	3
	.byte	4
\end{lstlisting}%
\lthtmlfigureZ
\lthtmlcheckvsize\clearpage}

{\newpage\clearpage
\lthtmlinlinemathA{tex2html_wrap_inline1048}%
$\backslash$%
\lthtmlinlinemathZ
\lthtmlcheckvsize\clearpage}

{\newpage\clearpage
\lthtmlinlinemathA{tex2html_wrap_inline1056}%
$\backslash\backslash$%
\lthtmlinlinemathZ
\lthtmlcheckvsize\clearpage}

\stepcounter{subsection}
\stepcounter{subsection}
\stepcounter{chapter}
\stepcounter{section}
\stepcounter{section}
\stepcounter{section}
\stepcounter{section}
\stepcounter{subsection}
\stepcounter{subsection}
{\newpage\clearpage
\lthtmlinlinemathA{tex2html_wrap_inline1058}%
$2 + 8 = 10$%
\lthtmlinlinemathZ
\lthtmlcheckvsize\clearpage}

\stepcounter{subsection}
\stepcounter{subsection}
\stepcounter{subsection}
\stepcounter{subsection}
\stepcounter{section}
\stepcounter{subsection}
\stepcounter{subsection}
\stepcounter{subsection}
\stepcounter{subsection}
\stepcounter{chapter}
\stepcounter{section}
\stepcounter{subsection}
\stepcounter{subsection}
\stepcounter{subsection}
\stepcounter{subsection}
\stepcounter{subsection}
\stepcounter{subsection}
\stepcounter{section}
\stepcounter{subsection}
\stepcounter{subsection}
\stepcounter{subsection}
\stepcounter{subsection}
\stepcounter{subsection}
\stepcounter{subsection}
\stepcounter{subsection}
\stepcounter{section}
\stepcounter{subsection}
\stepcounter{subsection}
\stepcounter{subsection}
\stepcounter{section}
\stepcounter{section}
\stepcounter{chapter}
\stepcounter{section}
\stepcounter{subsection}
{\newpage\clearpage
\lthtmlinlinemathA{tex2html_wrap_inline1060}%
$-1$%
\lthtmlinlinemathZ
\lthtmlcheckvsize\clearpage}

\stepcounter{subsection}

\end{document}
